\hypertarget{lecture-8---midterm-review}{%
\chapter{Lecture 8 - Midterm Review}\label{lecture-8---midterm-review}}

\hypertarget{general-exam-structure}{%
\subsubsection{General Exam Structure}\label{general-exam-structure}}

\begin{itemize}
\tightlist
\item
  She posted only calculations on Polylearn, these notes are her
  explaining it.
\item
  Can use R as your calculator on the computers.
\item
  Looks a lot like HW and DA. Calculations and stuff isn't that
  important, analysis is more important. WHY is more important. Real
  World conclusions!
\item
  Best way to study is to make up your own exam questions. She's gonna
  integrate one of the questions we submit onto the exam.
\end{itemize}

\hypertarget{question-1a}{%
\subsubsection{Question 1a}\label{question-1a}}

Some dealerships will price cars differently, not necessarily
independent (based on dealership preferences, might price in related
way)

Divide all covariances by constituent standard deviations (she used R
but you should be able to do this by hand)

Off diagonals show a slightly positive relationship. Toyota cars are
probably related, and negative relationships between Toyota and Ford
Fusion. Both hybrids? Anyway, positive relationship between first 2 and
negative with 3rd.

\[ \frac{3.1}{ \sqrt{7.1 \times 6.4}} \]

\hypertarget{question-1b}{%
\subsubsection{Question 1b}\label{question-1b}}

We're talking about a linear combination.

\[2y_1 + 4y_2 + 0y_3\]

\[a = (2 4 0)^T\]

\[\mu_{a'y} = a'\mu_y\]

\[\sigma_{a'y}^2 = a\Sigma_ya\]

She did this in R, but be able to do by hand.

\hypertarget{question-1c-hypothetical}{%
\subsubsection{Question 1(c,
hypothetical)}\label{question-1c-hypothetical}}

Distribution of how much Gates would spend?

Univariate normal: \[ \$ spent \sim N(142, \sqrt{177}) \]

\hypertarget{question-1d-hypothetical}{%
\subsubsection{Question 1(d,
hypothetical)}\label{question-1d-hypothetical}}

Probability that Gates spends more than \$150K?

Get Z score: \[Z = \frac{150-142}{\sqrt{177}}=0.6 \], get area above
that on a normal curve.

\hypertarget{question-2a}{%
\subsubsection{Question 2a}\label{question-2a}}

Fitness abilities are probably related for a given person.

\[H_0: \vec{\mu_L} = \vec{\mu_H}\] -- (Mean vector of low blood
pressures and high blood pressures are equal.)

\[H_a: \vec{\mu_L} \ne \vec{\mu_H}\] --\ldots{}not the same\ldots{}

We do not reject the null (that they have different fitness abilities).
We're given an F stat and p-value, so we fail to reject the null. Real
world conclusion: We do not have sufficient evidence to say that blood
pressure has an effect on fitness level.

\hypertarget{question-2b}{%
\subsubsection{Question 2b}\label{question-2b}}

\textbf{MVN?}

\begin{itemize}
\tightlist
\item
  Maybe, but there's only 10 subjects PER GROUP (Low group and High
  group), so we can't just assume multivariate normality based on
  Central Limit Theorem.
\item
  Also, we must have equal covariances because hotelling uses pooled
  sample covariance matrix and that relies on the assumption that we
  have equal covariances.
\item
  \textbf{TODO: WORKAROUND}: Not use pooled. \texttt{hotelling} package
  doesn't do that though.
\item
  Recall \[ S_{\bar{p}} = \frac{S_1}{n_1} + \frac{S_2}{n_2} \]
\item
  People are all independent from each other (assuming no family
  members).
\end{itemize}

\hypertarget{question-3a}{%
\subsubsection{Question 3a}\label{question-3a}}

\begin{itemize}
\tightlist
\item
  Recall sample total variances (\[ \sum diag(S) \]) and sample
  generalized variances (\[ | S |\]) from HW1
\item
  Total variance: Similar, little more uncertainty in the uptake rate in
  the chilled sitaution.
\item
  Generalized variance: Maybe relationship between concentration and
  rate of uptake is not the same between chilled and unchilled. Little
  more overall uncertainty in the chilled case.
\end{itemize}

\hypertarget{question-3b}{%
\subsubsection{Question 3b}\label{question-3b}}

\emph{This question is phrased as it would be on the midterm!}

\[H_0: \vec{\mu}_{chilled} = \vec{\mu}_{non-chilled}\]

\[H_a: \vec{\mu}_{chilled} \ne \vec{\mu}_{non-chilled}\]

\[\bar{d} = \bar{y_1} - \bar{y_2}\]

Calculate as non-pooled, this question is ambiguous but the midterm will
be more clear.

\[S_{\bar{p}} = \frac{S_1}{n_1} + \frac{S_2}{n_2}\]

We care about our \[T^2\] statistic and how variable it is:

\[T^2 = \bar{d}'(S_{\bar{d}}^-1)\bar{d} = 10.3\]

10.3 \textgreater{} critical value of 6.29 (we need to know if above or
below), so we reject the null hypothesis. Conclusion: The mean vectors
between chilled and non-chilled are not equal, so \textbf{we have found
evidence at the} \[\alpha = 0.05\] \textbf{level that chilled
environments affect} \[CO_2\] \textbf{concentration and uptake}.

\hypertarget{question-4a}{%
\subsubsection{Question 4a}\label{question-4a}}

Do they all have the same HWY MPG, engine size, and horsepower?

\[H0: \vec{\mu}{sports} = \vec{\mu}{SUV} = \vec{\mu}{wagon}\]

\[H_a\]: At least one mean is different

(Extra: give a model and a null hypothesis in terms of the model).

Model:

\[y = \mu + \alpha_j + \epsilon_{ij}\]

Recall in English, \[y_{ij}\] is car \[i\] of type \[j\].

\[H_0: \forall_j \alpha_j = 0\]

Using Wilks' Lambda:

\[\Lambda = \frac{| E |}{| E + H|}\]

which we expect to be 1 under the null. Actual value is 0.2849\ldots{}
which is \textbf{not} close enough to 1 (not greater than given critical
value of 0.90). Therefore, we reject the null hypothesis and
\textbf{conclude that the three cars do NOT have the same Highway MPG,
Engine Size, and Horsepower.}

Using Pillai, you should be able to explain what you expect under the
null.

\hypertarget{question-4b}{%
\subsubsection{Question 4b}\label{question-4b}}

\textbf{What were the assumptions?}

\begin{itemize}
\tightlist
\item
  MVN -- Sample sizes are pretty high, so yes.
\item
  Homogenous (equal) covariance matrices -- Just by eyeballing,
  determinants are not the same.
\end{itemize}

\hypertarget{question-4c}{%
\subsubsection{Question 4c}\label{question-4c}}

\[H0: \Sigma_S = \Sigma_W = \Sigma{SUV}\]

\[H_a\]: Not all equal.

Test statistic: Box's M Test. We have 3 populations.

\[M = \frac{(4.98 \times 10^4)^{46/2}(2.1 \times 10^4)^{58/2}(3.17 \times 10^4)^{38/2}}{(4.53 \times 10^4)^{(46+58+33)/2}} = 1.1 \times 10^-11\]

M value is very small, so we reject null because this is a ratio, and if
it's not close to 1 then it implies that the individual matrices don't
equal the Pooled version. This means we can't really trust the results
of Wilks' Lambda and the outputs of the other tests because our basic
assumptions are not valid.

\hypertarget{question-5a}{%
\subsubsection{Question 5a}\label{question-5a}}

Are video games related at all to health?

\[H_0: \Sigma = \Sigma_0 \] (means this is unknown, and our best guess
about it is the sample version)

\[H_a: \Sigma = diag(\sigma_n^2) \] with 0s everywhere else.

Therefore we're testing if \[ | \Sigma | = \sigma1^2 | \Sigma{23} | \]

Test statistic:

\[u = \frac{|S|}{S_1^2 |S_{W, BP}} = 0.99\]

Note that \[S_1^2\] is the variance of the first variable (value is
8.31, top left in matrix).

We reject the null hypothesis and conclude that gaming is independent of
Blood Pressure and Weight.

\hypertarget{homework-review-part-ii}{%
\subsubsection{Homework Review, part II}\label{homework-review-part-ii}}

\hypertarget{question-1}{%
\paragraph{Question 1}\label{question-1}}

\[y_{ijk} = \mu + \alpha_j + \beta_k + \Gamma_{jk} + \epsilon_{ijk}\]

where \[y\] is the observation of those 4 measurements (4x1 vector),
\[i\] is the individual plant, \[j\] is the variety, and \[k\] is the
date.

\textbf{3 Null Hypotheses to test}:

1) \[H_0: \alpha_j = 0, \Lambda = 0.9 -> p = 0.27\] so fail to reject.

2) \[H_0: \beta_k = 0, \Lambda = 0.034 -> p = 0\] so reject.

3) \[H_0: \Gamma_{jk} = 0, \Lambda = 0.94 -> p = 0.51\] so fail to
reject.

\textbf{Sowing date affects the mean of 4 growth measures. We do not
have evidence to show that variety matters, or that variety modifies the
effect of sowing date (interaction term).}

\hypertarget{question-2}{%
\paragraph{Question 2}\label{question-2}}

Are the assumptions met?

We did not find evidence of homogeniety of covariance matrices, so we
cannot verify our evidence.
