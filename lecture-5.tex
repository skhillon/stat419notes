\hypertarget{lecture-5---thu-apr-18}{%
\chapter{Lecture 5 - Thu, Apr 18}\label{lecture-5---thu-apr-18}}

\hypertarget{testing-covariance-matrices-by-determinants}{%
\paragraph{Testing Covariance Matrices by
Determinants}\label{testing-covariance-matrices-by-determinants}}

How do you find determinant of 4x4 matrix?

\[A = \left(\begin{array}{cccc} 16 & 0 & 0 & 0 \\ 0 & 1 & 2 & 0 \\ 0 & 3 & 4 & 0 \\ 0 & 0 & 0 & 74 \end{array}\right)\]

\[A\] is called a \emph{block-diagonal matrix}. There are ``squares''
down the diagonal line. First block is 16, second block has rows {[}1,
2{]} and {[}3, 4{]}, and third block is 74.

You can find \[| A | = | a_1 |  | a_2 |  | a_3 | = 16(-2)(74) = -2,368\]

\textbf{College Admisssions}

Let's say you had a matrix about applying to college. You have a vector
with the following values:

\[\left(\begin{array}{c} GPA \\ SAT \\ ACT \\ RecLetter1 \\ RecLetter2 \end{array} \right)\]

The numerical ones are GPA, SAT, and ACT. The ``Holistic'' ones are
RecLetter1 and RecLetter2 for rec letters.

We want 2 subvectors, \[\overrightarrow{x}\] and \[\overrightarrow{y}\].
Is it true that rec letters are capturing the same capabilities as
numerical? They should all capture the same thing: student quality.

We now have:

\[\left(\begin{array}{c} x_1 \\ x_2 \\ x_3 \\ y_1 \\ y_2 \end{array} \right)\]

There is some true covariance matrix, let's call it \[\Sigma\], of these
variables. The research question you might have is, are these holistic
measures independent of these numerical measures? (by independent in
this class, we mean ``uncorrelated'', or that the true correlation is 0)

\textbf{If correlation is 0 (under} \[H_0\] \textbf{that these are
unrelated), what does} \[\Sigma\] \textbf{look like?}

\[H0: \Sigma = \left(\begin{array}{c} \sigma{11}^2 & \sigma \\ \sigma & \sigma_{55}^2 \ \end{array} \right) = \Sigma_0\].
Note that this is really a 5x5, we just skipped to the ends.

\[H_a: \Sigma = \left(\begin{array}{c} \Sigma_Y & 0 \\ 0 & \Sigma_X \ \end{array} \right)\]

Under \[H_a$, $| \Sigma | = | \Sigma_x | | \Sigma_y |\]

We've calculated some overall \[S\], a sample covariance matrix:

\[S = \left(\begin{array}{c} Sy & S{xy} \\ S_{xy} & S_x \ \end{array} \right)\]

What is our test statistic? Should be ratio of determinants: we have
\[\frac{| S |}{| S_x | |S_y|}\], should be close to 1 (up to your
judgement). It doesn't follow any distribution, bounded between 0 and 1,
represents correlation. If you got 0.8, then you could say that they
both represent similar things about student performance.

\textbf{Are all 5 of these measures independent from each other?}

\[H_0\] remains the same as before.

\[H_a\] is still \[\Sigma\], but is now a diagonal matrix, where diag is
variances of each variable. Therefore, determinant is the product of the
diagonals, which is the product of all variances (estimated variances
from the data). We still want this to be close to 1.

We somehow got that GPA is independent from SAT/ACT and they're
independent from Rec letters. TODO

\textbf{We think variances are all equal, and they are all completely
independent.}

\[H_0\] stays the same.

We make \[H_a\] a diagonal matrix of the same variance \[\sigma^2\]. So
determinant = multiplying them all together.
\[ | H_a | = (\sigma^2)^p \], where \[p\] is number of elements.

What is the test statistic? This time it's changed. We have
\[\frac{| S |}{(S_{pooled}^2)^5}\]

which we still want it to be close to 1.

\textbf{Switching things up}

\begin{itemize}
\tightlist
\item
  Drop GPA from the study.
\item
  Our assertion is that numeric variables are independent of rec
  letters, and that their covariances are the same.
\end{itemize}

\[H_0: \Sigma = \Sigma_0\]

\[H_a: \Sigma\] is a block diagonal

\[\Sigma = \left(\begin{array}{c} \Sigma_Y & 0 \\ 0 & \Sigma_X \\ \end{array} \right)\]

And the test statistic is once again \[\frac{| S |}{| S_Y | | S_X |}\]

\textbf{Is the mean vector for rec letters the same as the mean vector
for test scores?}

\begin{itemize}
\tightlist
\item
  We want to pool covariance matrices.
  \[H_0: \overrightarrow{\mu_{rec}} = \overrightarrow{\mu_{scores}}\]
\end{itemize}

Then we have \[M\]

\begin{itemize}
\tightlist
\item
  On exams we're not getting anything on a formula sheet.
\item
  Box's M stat is given to us, the one above.
\end{itemize}
