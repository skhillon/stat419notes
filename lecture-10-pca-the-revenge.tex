\hypertarget{lecture-10---pca-the-revenge}{%
\chapter{Lecture 10 - PCA: The
Revenge}\label{lecture-10---pca-the-revenge}}

Recall that a PC is a linear combination of variables that has the
largest variance.

Suppose you have some vector \[\vec{a}\]. We're going to take this and
multiply it by our data: \[z = \vec{a}\vec{y}\]. The variance is
\[var(z) = \vec{a}'\Sigma_y\vec{a}\]. What \[\vec{a}\] maximizes the
variance? Not a trick question, it's \[(\infty, \infty)'\].

But that's not useful. We want to stay within our data, so we scale it.
We want to find an \[\vec{a}\] that maximizes \[\lambda\] such that:

\[ \lambda = \frac{\vec{a}'S_y\vec{a}}{\vec{a}'\vec{a}} \]

We can rearrange terms to get the following equation:

\[\vec{a}'(I\lambda - S_y)\vec{a} = 0\]

Note the part in parentheses. Recall that solving
\[| I\lambda - S_y | = 0\] finds the eigenvalues of the covariance
matrix by solving for \[\lambda\]. We want the best \[\vec{a}\] that
blows up the variance \[\lambda\] without blowing up itself.

In the equation solving for 0 above
{[}\[\vec{a}'(I\lambda - S_y)\vec{a} = 0\]{]}, there will be multiple
solutions; in fact, there will be \[p\] solutions matching \[p\]
variables. We sort the variables from largest to smallest and denote
them as \[a_1, ..., a_p\] and \[\lambda_1, ..., \lambda_p\]. \[a_i\] are
your Principal Components, and \[\lambda_i\] are the relative variance
that you're trying to maximize. Lastly, your Principle Component Scores
are \[z_1 = a_1'\vec{y}, z_2 = a_2'\vec{y}, ..., z_p = a_p'\vec{y}\].

\textbf{NOTE}: Why are we doing this? We're trying to find a linear
combination that spreads the data as much as possible. Why? Depends on
your research question, but in general you want to differentiate
``good'' from ``bad'' scores as much as possible so that it's clear.

\textbf{NOTE}: On homework, note that ``first PC'' just the ``best
ranking method''

Lastly, we said PCA was unsupervised; you're trying to score all your
students. The natural next question is, can we decide what's the most
important linear combination if we have some extra information,
e.g.~supervised method? Yes, this is called Linear Discriminant
Analysis.

\hypertarget{linear-discriminant-analysis-lda}{%
\subsubsection{Linear Discriminant Analysis
(LDA)}\label{linear-discriminant-analysis-lda}}

NOTE: This is NOT Latent something else, LDA is usually used to
reference the other one.

This is a supervised method; we know which population each observation
came from. We want to know: what linear combination \[\vec{a}'\vec{y}\]
best separates your populations? (Ex: What weightings of survey answers
will best differentiate between favorite students?)

\textbf{The best linear combination is the one that provides the most
evidence of a difference.} For example:

\[H_0: \vec{a}'\vec{\mu_1} = \vec{a}'\vec{\mu_2}\] (Does applying these
principle components make them equal? We want to reject).

\[H_a: \vec{a}'\vec{\mu_1} \ne \vec{a}'\vec{\mu_2}\]

We also observe \[Y_1, Y_2\] and we observe \[\Sigma_1 = \Sigma_2\].
This will be an ordinary t test because the linear combination
\[\vec{a}'\vec{y}\] results in a scalar.

The evidence of what you observe is the difference
\[\vec{a}'\bar{y_1} - \vec{a}'\bar{y_2}\]. Under the null, you would
expect a difference of \[0\] if they're equal. We find the pooled
covariance matrix \[S_p\] as follows:

\[(\frac{1}{n_1} + \frac{1}{n_2})S_p = S_\bar{d}\]

\[S_z = \vec{a}'S_p\vec{a}\] where \[z = \vec{a}'y_1 - \vec{a}'y_2\]

The t statistic (want to maximize magnitude to find the most evidence!)
is then:

\[t = \frac{\vec{a}'\bar{y_1} - \vec{a}'\bar{y_2}}{\sqrt{\vec{a}'S_p\vec{a}(\frac{1}{n_1} + \frac{1}{n_2})}}\]

We find the most evidence when we maximize the following (square t
statistic from above, take out sample sizes since those are constants):

\[\frac{(\vec{a}'\bar{y_1} - \vec{a}'\bar{y_2})}{\vec{a}'S_p\vec{a}}\]

If you think about what's actually happening here (divide out the pooled
covaraince that has been adjusted), note that dividing by a matrix
implies some sort of inverse. Skipping all the math, you end up with
this:

\[\vec{a} = S_{p_{(p \times p)}}^{-1}(\bar{y_1} - \bar{y_2})_{(p \times 1)}\]

This equation will give you the coefficients (Principal Components) that
will maximize your difference.

Punchline:
\[\vec{a}'\bar{y_1} - \vec{a}'\bar{y_2} = (\bar{y_1} - \bar{y_2})'S_p^{-1}(\bar{y_1} - \bar{y_2})\]

What is the right half of the equation? \textbf{Hotelling's T-Squared
(without sample size)}!

\[\vec{a}'\bar{y_1} - \vec{a}'\bar{y_2} = T^2(\frac{1}{n_1} + \frac{1}{n_2})^{-1}\]

**The \[T^2\] test is \emph{exactly} finding a linear combination to
best separate the populations and then testing to see if the populations
were separated enough for statistical significance.**

\hypertarget{small-adjustment}{%
\paragraph{Small Adjustment}\label{small-adjustment}}

Suppose you have 3 populations, and you need to find 2 linear
populations that best separate the 3 populations. We then use
Hotelling-Lawley.

**Note that coefficients of a linear discriminant are \[S_p\], and you
may be asked to calculate that on an exam.**

\hypertarget{why-find-linear-discriminants}{%
\subsubsection{Why Find Linear
Discriminants?}\label{why-find-linear-discriminants}}

With PCA you want to find the variable that best describes the data.
With LDA, you can ask:

\begin{enumerate}
\def\labelenumi{\arabic{enumi}.}
\tightlist
\item
  Which coefficients tell you the relative importance of variables in
  distinguishing between individual populations?
\item
  From that, find the least important (redundant) variables and drop
  them --\textgreater{} Dimensionality Reduction.
\end{enumerate}

\hypertarget{testing-for-redundant-variables}{%
\paragraph{Testing for Redundant
Variables}\label{testing-for-redundant-variables}}

Let's say we think we don't need the last few variables to differentiate
between populations in the data (our main goal in this whole thing). We
test:

\[H_0: \vec{a}'\bar{\mu_1} = \vec{a}'\bar{\mu_2}\]

\[H_a: \vec{a}'\bar{\mu_1} \ne \vec{a}'\bar{\mu_2}\]

We basically do a full-vs-reduced test, where the reduced is filled with
\[0\]s to keep vectors the same:

Full: \[\vec{a} = (a_1, a_2, ..., a_p)\]

Reduced: \[\vec{a} = (a_1, a_2, ..., 0, 0, 0)\]
