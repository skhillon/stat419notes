\hypertarget{lecture-4---tue-apr-16}{%
\chapter{Lecture 4 - Tue, Apr 16}\label{lecture-4---tue-apr-16}}

Tuesday, Apr 16

Review: Pooled covariance matrix of \[d\]:
\[C = \frac{(n_1 - 1)C_1 + (n_2 - 1)C_2}{n_1 + n_2 - 2}\]

Covariance estimate for mean dif:
\[S_\bar{d} = (\frac{1}{n_1} + \frac{1}{n_2})C\]

\[C^{-1}(\frac{n_1 + n_2}{n_1n_2})^{-1} \to (\frac{n_1 + n_2}{n_1n_2})C^{-1}\]

\[T_2 = \bar{d}^{-1} S_\bar{d}^{-1}\bar{d} = (\frac{n_1 + n_2}{n_1n_2})C^{-1}\]

\hypertarget{checking-multivariate-normality-mvn}{%
\paragraph{Checking Multivariate Normality
(MVN)}\label{checking-multivariate-normality-mvn}}

\begin{enumerate}
\def\labelenumi{\arabic{enumi}.}
\tightlist
\item
  Individual Variables
\end{enumerate}

\begin{itemize}
\tightlist
\item
  Histograms
\item
  Q-Q plots.
\item
  What is a Q-Q plot? AKA Normal Quantile Plot.
\item
  If a histogram is normal, how many standard deviations do we have to
  go below the mean before there's 16\% of the data? Should be 1
  standard deviation. Mark where that happens.
\item
  We can keep finding checkpoints where a certain percentage of data is
  captured.
\item
  We look at what should happen on a standard normal and what happens
  with my data. On a standard normal, 0 standard deviations from mean
  should have half the data below, but our data showed 67\%. At Z-scores
  of 1 and 2, we see values of 71 and 75.
\item
  How much do these values fall on a nice normal distribution? Plot X
  being normal values, Y being our actual values. Q-Q plot shows a
  linear plot, how well your values fit the line is how normal your data
  is.
\item
  Recall 324, top left chart!
\item
  Shapiro-Wilk: Basically tests quantiles from Q-Q plot. Gives you
  something better than eyeballing a plot for linearity.
\end{itemize}

\begin{enumerate}
\def\labelenumi{\arabic{enumi}.}
\setcounter{enumi}{1}
\tightlist
\item
  MVN: Looking at Quantiles for variance and ``Kirtosis'' (don't worry
  about it).
\end{enumerate}

\begin{itemize}
\tightlist
\item
  Command shows output of a bunch of tests, don't worry about any that
  aren't in these notes.
\item
  Central Limit Theorem trumps all, sample size (\[n\] \textgreater{}
  15) is large enough to treat as normal.
\item
  Sometimes \[|\Sigma| \ne 0\] happens
\end{itemize}

\hypertarget{hypothesis-test-for-sugar-percentage}{%
\paragraph{Hypothesis Test for Sugar
Percentage}\label{hypothesis-test-for-sugar-percentage}}

\[H_0: \mu{ResSug} = 0.8\mu_{pH}\]

\[\overrightarrow{\mu} = \left(\begin{array}{c} \mu{ResSug} \\ \mu{pH} \\ \end{array} \right)\]

Note that if you change the order where \[pH\] is first (as was the case
in the R code for Lab 3 before changing it), you need to switch \[-0.8\]
to be first (on top).

\[\mathbf{a}\overrightarrow{\mu}\] where
\[ \mathbf{a} = \left( \begin{array}{c} 1 \\ 0.8 \\ \end{array} \right)\]

\[\mathbf{w} = \mathbf{a'y}\]

\[\mathbf{W} = \mathbf{Ya}\]

\hypertarget{hypothesis-test-for-generalized-variance}{%
\paragraph{Hypothesis Test for Generalized
Variance}\label{hypothesis-test-for-generalized-variance}}

\[H_0: \Sigma_g = \Sigma_b (= \Sigma_p)\]

\[\frac{| \Sigma_g |}{| \Sigma_b |} = 1\]

This shows that the ratio of determinants to see if covariance matrices
are equal: \[\frac{ \sqrt{|\Sigma_g||\Sigma_b| }}{|\Sigma_p|} = 1\]

The following should be close to 1:
\[\mu = \frac{|S_g|^{\frac{1}{2}(n_1 - 1)}|S_b|^{\frac{1}{2}(n_1 - 1)}}{|S_p^2|^{(n_1 + n_2 - 2)}}\]

\hypertarget{individual-t-tests}{%
\paragraph{Individual T-Tests}\label{individual-t-tests}}

Our tests are on vectors, but we don't really know which ones are
driving the differences. We can run individual t tests using individual
calls to \texttt{t.test}.

\emph{Question: Why not just start here? If one of them is different
then we don't have to bother with the matrix stuff, right?}

Answer: We're looking at relationships between variables. pH and
residual sugar of the same wine are linked. In the Hotelling test, we
accounted for that relationship to test the vectors all at once. That's
why getting to \texttt{S\_dbar} was such a pain because we care about
the covariances.

Note that individual tests can be significant whereas the multivariate
tests can be not significant, or vice-versa. The relationship matters,
so you have to run both depending on what you want.
