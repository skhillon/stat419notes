\hypertarget{lecture-7---thu-apr-25}{%
\chapter{Lecture 7 - Thu, Apr 25}\label{lecture-7---thu-apr-25}}

\emph{Note: She put up practice midterm questions and she's going to be
typing up notes on a flight sometime this weekend.}

\hypertarget{faqs}{%
\subsubsection{3 FAQs}\label{faqs}}

\textbf{1 - Pooling}

\begin{itemize}
\tightlist
\item
  There's some true covariance matrices which we don't know
  (\[\sigma_1\] and \[\sigma_2\]) which we want to estimate.
\item
  If we believe the two covariance matrices are equal (pooled), then our
  best guess is that they're both equal to \[S_p\]
\item
  If we believe they are not equal, then we guess \[S_1, S_2\]
\item
  Use Box's M test to determine whether or not to pool (see if the two
  sigmas are equal).
\end{itemize}

\[\Sigma_{\bar{d}} = \Sigma_{\bar{y_1}} + \Sigma_{\bar{y_2}} = \frac{\Sigma_1}{n_1} + \frac{\Sigma_2}{n_2}\]
(adjusting for sample size by dividing).

What's our best guess about \[ \Sigma_{\bar{d}} \] ?

If pooling, then:

\[S_{\bar{d}} = \frac{S_p}{n_1} + \frac{S_p}{n_2} = (\frac{1}{n_1} + \frac{1}{n_2})S_p\]

Else if not pooling, then:

\[S_{\bar{d}} = \frac{S_1}{n_1} + \frac{S_2}{n_2}\]

\textbf{2 - Null/Alternate Hypotheses}

\begin{itemize}
\tightlist
\item
  Based on research question
\item
  ``Is there evidence that\ldots{}'' (ex: ``\ldots{}all variables are
  independent?'' or ``\ldots{}cov matrices are not equal?'' based on
  Box's M test)
\item
  Ex: \[ H_0: \Sigma \] or \[ \Sigma_1 = \Sigma_2 \] (based on pooling?)
\item
  Ex: \[ H_a: \Sigma = diag(\sigma^2) \]
\end{itemize}

\textbf{3 - Test statistic to p-value}

\begin{itemize}
\tightlist
\item
  We have a bunch of test statistics: \[T^2, \Lambda, V, U, M, u\]
\item
  All of these turn into a p-value, which is how we make our final
  conclusion.
\item
  The intermediate step (not very important, will be given to you in R
  output):
\item
  Compare to F distribution: \[T^2, \Lambda, V, U \]
\item
  Compare to \[ \chi^2 \] distribution: \[M, u\]
\item
  Know how to conclude based on hypotheses and \textbf{be able to
  discuss these, what it means}
\end{itemize}

\hypertarget{reviewclarify-manova}{%
\subsubsection{Review/Clarify MANOVA}\label{reviewclarify-manova}}

\begin{itemize}
\tightlist
\item
  MANOVA only works if pooled (that \[\Sigma\] matrices are equal)
\item
  If we want to test for quality of all 3 covariance matrices, we can
  just add another to the top and change \[p\] on the bottom to 3.
\end{itemize}

\hypertarget{new-two-way-manova}{%
\subsubsection{New: Two-way MANOVA}\label{new-two-way-manova}}

\begin{itemize}
\tightlist
\item
  Let's say, in addition to collecting ACT and SAT from these 3 schools,
  we also want to collect whether that student is in-state or
  out-of-state.
\end{itemize}

\hypertarget{review-how-would-you-test-the-following-null-hypotheses}{%
\paragraph{Review: How would you test the following null
hypotheses?}\label{review-how-would-you-test-the-following-null-hypotheses}}

Can we say that\ldots{}

\textbf{1)} \[H_0: \frac{SAT}{68} = ACT\] \textbf{?}

Guess: Compare means same

Ans:

\begin{itemize}
\tightlist
\item
  \[H_0: \vec{a}'\vec{\mu} = 0\]
\item
  t-test of linear combination of means: \[ \vec{a} = (1/68, -1)\]where
  first is SAT/68 and second is ACT.
\end{itemize}

\textbf{2)} \[H_0:\] \textbf{SAT is independent of ACT? (lazy notation)}

Guess: Calculate a Correlation statistic

Ans: Test statistic is Ratio of determinants and
\[ S = diag(S_1^2, S_2^2) \]

\textbf{3)} \[H_0\]\textbf{: Same test scores for in-state vs
out-of-state students?}

Guess: Mean in-state same as mean out-of-state

Ans: Calculate \[T^2\] statistic (can still do that with 2 groups), same
assumptions as num 4.

\textbf{4)} \[H_0\]\textbf{: Same test scores across all 3 schools?}

Guess: Compare 3 means

Ans: MANOVA test statistics guessing equality of mean vectors across 3
populations.

Assumptions:

\begin{enumerate}
\def\labelenumi{\arabic{enumi}.}
\item
  That you can pool (equal covariance matrices)
\item
  Random samples (truly independent samples)
\item
  Multivariate Normality (hope Central Limit Theorem applies and move
  on)
\end{enumerate}

\textbf{5)} \[H_0\]\textbf{: Test scores same regardless of school and
in/out-of-state status?}

Guess: Compare in-state from all schools vs out-of-state from all
schools.

Ans: 2-way MANOVA

\[H_0: \mu_1 \] (TODO: ???)

Use \[i\] for student, \[j\] for school, and \[k\] for categorical (0
for in-state, 1 for out-of-state)

You say there's one mean + some random noise:
\[ \vec{y_{ijk}} = \vec{\mu} + \varepsilon_{ijk} \].

\[H_A\] and \[H_B\] for categorical variable \[A\] and categorical
variable \[B\]. This is \textbf{not} null vs alternative. A is in-state,
B is out-of-state.

Sources of error:

\begin{enumerate}
\def\labelenumi{\arabic{enumi}.}
\tightlist
\item
  \[H_A\]: There may be different means between schools.
\item
  \[H_B\]: There could be error because we didn't account for different
  in- vs out-of-state means (TODO??)
\item
  \[H_{AB}\]: Maybe in- vs out-of-state matters differently for each
  school. Maybe out-of-state students who choose to go to CSU
  Bakersfield are weird, but going out-of-state to UCSB is normal so
  there's not much difference between in- and out-of-state students at
  UCSB. We're looking at different \textbf{subgroupings} across all of
  these subgroups. Equation for this doesn't matter.
\item
  Noise (\[E\]): Even after accounting for all of this, there's still
  some unexplained error which you just have to live with.
\end{enumerate}

\hypertarget{our-model-is-now}{%
\paragraph{Our model is now}\label{our-model-is-now}}

Our model (before removing unnecessary adjustments) is now:

\[\vec{y_{ijk}} = \vec{\mu} + \vec{\alpha_j} + \vec{B_c} + \vec{\gamma_{jk}} + \vec{E_{ijk}}\]

where:

\[\vec{\mu}\] is the overall mean.

\[\vec{\alpha_j}\] is the adjustment you need to make for each school
(if Cal Poly regularly gets 200 higher than others then
\[\alpha_j = 200\] for Cal Poly). Source of error 1.

\[\vec{B_c}\] is the adjustment for in- vs out-of-state. (source of
error 2)

\[\vec{\gamma_{jk}}\] is a ``tweak'' for interactions between school
(source of error 3)

and \[\vec{E_{ijk}}\] is the leftover random noise (source of error 4)

We want to compare model with \[\vec{H_{AB}} vs \vec{E}\]. We test this
using Wilk's test:

\[\Lambda = \frac{| \vec{E} |}{| \vec{E} + \vec{H_{AB}} |}\]

We also test: should we have to keep \[ \vec{\alpha_j} \] in the model?
(another \[H_0\], where we test \[\vec{H_A} vs \vec{E}\]):

\[\Lambda = \frac{| \vec{E} |}{| \vec{E} + \vec{H_{A}} |}\]

And additionally: should we have to keep \[ \vec{\alpha_j} \] in the
model? (another \[H_0\] where we test \[\vec{H_B} vs \vec{E}\]):

\[\Lambda = \frac{| \vec{E} |}{| \vec{E} + \vec{H_{B}} |}\]

Looking at Wilks, we have to keep the factor for schools
(\[\vec{\alpha_j}\]), but we don't have to keep track of in- vs
out-of-state or any interaction terms, so our final model is:

\[\vec{y_{ijk}} = \vec{\mu} + \vec{\alpha_j} + \vec{E_{ijk}}\]
